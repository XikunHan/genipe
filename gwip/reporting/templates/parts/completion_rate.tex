
To evaluate the completion rate, we first used a probability threshold of
$\geq \VAR{ prob_threshold }$\%, which means that a genotype must have one of
the three allele combination (\texttt{AA}, \texttt{AB} or \texttt{BB})
probabilities higher or equal to \VAR{ prob_threshold }\% to be considered as a
\textit{good call}.\\

For the \VAR{ nb_imputed } imputed variants, an average completion rate of
\VAR{ average_comp_rate }\% was obtained. When removing variants with a
completion rate under \VAR{ rate_threshold }\%, \VAR{ nb_good_sites }
(\VAR{ pct_good_sites }\%) markers were left, with an average completion rate
of \VAR{ average_comp_rate_cleaned }\%, meaning that there is a mean of
\VAR{ mean_missing } missing genotypes (for \VAR{ nb_samples } samples) for
each markers.\\

A total of \VAR{ nb_genotyped } variants were previously genotyped,
\VAR{ nb_genotyped_not_complete } (\VAR{ pct_genotyped_not_complete }\%) of
which had  a call rate lower than 100\%. A total of
\VAR{ nb_geno_now_complete } (\VAR{ pct_geno_now_complete }\%) missing
genotypes were imputed with high quality (\textit{i.e.}
\VAR{ nb_site_now_complete } markers now have a call rate of 100\%).

